% ========================================
% FOOTNOTES AND MARGIN NOTES
% Options and packages for foot and margin rules
% ========================================


% CODE DISPLAYING
% The verbatim environment enables to display code safely:
% - Its starred version verbatim* also prints spaces
% - The verb command has the same effect as verbatim within a paragraph


% CODE HIGHLIGHTING
\usepackage{minted}

% The minted environment enables to highlight code:
% - Argument: name of the programming language
% - Additional parameters: frame, framesep, baselinestretch, bgcolor, fontsize, linenos, mathescape, rulecolor, showspaces
% - example: \begin{minted}[frame=lines, framesep=2mm, baselinestretch=1.2, bgcolor=LightGray, fontsize=\footnotesize, linenos]{python} ... \end{minted}

% The \inputminted can input code from a file:
% - First argument: programming language
% - Second argument: file name
% - Optional parameter: firstline, lastline
% - Example: \inputminted[firstline=2, lastline=12]{octave}{BitXorMatrix.m}

% Inline code can be added in a paragraph with the \mint{<language>}|<code>| command.


% CODE LISTING (recommended to use minted rather)
\usepackage{listings}
% The ltlistings environment enables to display code with language specific standard highlighting:
% - Optional parameters: language, caption
% - Example: \begin{lstlisting}[language=Python] ... \end{lstlisting}

% The lstinputlisting command enables to display code from a file:
% - Optional parameters: language, firstline, lastline
% - Compulsory argument: name of the file to use
% - Example: \lstinputlisting[language=Octave, firstline=2, lastline=12]{BitXorMatrix.m}

% Styles can be applied to color code listings:
% - Several examples/templates can be found online
% - More details: https://fr.overleaf.com/learn/latex/Code_listing

% A list of listings can be added in the document with the command \lstlistoflistings