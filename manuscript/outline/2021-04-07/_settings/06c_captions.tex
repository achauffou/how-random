% ========================================
% CAPTIONS
% Figures and tables captions
% ========================================

% CAPTION PACKAGE
% The caption and subcaption packages help to format captions of figures and subfigure. It offers several options to overwrite default LaTeX behaviours. Details available in the package manual.
\usepackage{caption}
\usepackage{subcaption}

% CAPTION SEPARATORS
% There are several caption separators available, including none, colon, period, space, quad, newline, endash.
% To add other caption separators, use the \DeclareCaptionLabelSeparator{<name>}{<separator>} command:
\DeclareCaptionLabelSeparator{bar}{ | }

% CAPTION COLORS
% Colors to use for caption label, text or both can be declared here:
\DeclareCaptionFont{red}{\color{red}}
\DeclareCaptionFont{green}{\color{green}}
\DeclareCaptionFont{blue}{\color{blue}}

% OTHER OPTIONS
% Most options can be tweaked. Refer to the package manual to see how to tweak them.

% CAPTION SETUP
% Use the \captionsetup[<environment>]{<options>} command at any time to set options for captions formatting. Specific environments can be figure, table, subfigure. Several options are available including:
% - format: plain, hang, ...
% - indentation: indentation of line starting at second line of the caption
% - labelformat: original, empty, simple, brace, parens
% - labelsep: none, colon, period, space, quad, newline, endash (others can be added manually, see above)
% - textformat: empty, simple, format
% - justification: justified, centering, centerlast, centerfirst, raggedright, Raggedright, raggedleft
% - singlelinecheck: true (automatically center caption if single line), false
% - font, labelfont, textfont: many options including size names, two-letters text format names, color=<color> (must be declared; see above)
% - font+=, label+=, textfont+=: font options to add to current ones
% - margin, width, oneside, twoside, margin*=, minmargin, maxargin, parskip, hangindent
% - style: predefined styles (base, default defined by caption package; additional can be declared)
% - skip: vertical spacing between caption and figure/table
% - position: assumed position of the caption (for space placement between caption and figure/table)
% \captionsetup[figure]{labelfont=bf,labelsep=bar,font=small,position=below}
% \captionsetup[table]{labelfont=bf,labelsep=bar,font=small,position=below}

% CAPTION FOR LIST OF FIGURE AND TABLES
% To modify the caption to use in the list of figures or tables, use an optional argument in the \caption command:
% \caption[<list entry>]{<heading>}.

% Using an empty option will remove the figure/table from the list of figures/tables:
% \caption[]{A figure without list entry.}

% UNNUMBERED CAPTION
% Use \caption* to include a caption without number and that will not appear in the list of captions.

% CAPTION OUTSIDE FLOATING ENVIRONMENT
% To add a caption outside a floating environment, use the commands \captionof or \captionof*:
% \captionof{<float type>}[<list entry>]{<heading>}

% ADD CAPTION ENTRY IN THE LIST OF FIGURES/TABLES
% To add manually a caption to the list of figures/tables, use \captionlistentry.
\captionsetup[table]{labelfont=bf,labelsep=bar,font=small,position=below,justification=raggedright,singlelinecheck=false}
